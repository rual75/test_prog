\hypertarget{application_8hpp}{
\section{Файл D:/Project/Cortex/test\_\-prog/Src/application.hpp}
\label{application_8hpp}\index{D:/Project/Cortex/test\_\-prog/Src/application.hpp@{D:/Project/Cortex/test\_\-prog/Src/application.hpp}}
}
: содержит класс, использующий STL, и функцию для распознования команды в строке  


{\tt \#include \char`\"{}main.h\char`\"{}}\par
{\tt \#include \char`\"{}cmsis\_\-os.h\char`\"{}}\par
{\tt \#include \char`\"{}stm32f3xx\_\-hal.h\char`\"{}}\par
{\tt \#include $<$strstream$>$}\par
{\tt \#include $<$string$>$}\par
{\tt \#include $<$map$>$}\par
{\tt \#include $<$string.h$>$}\par
\subsection*{Классы}
\begin{CompactItemize}
\item 
class \hyperlink{class_command_parsing}{CommandParsing}
\end{CompactItemize}
\subsection*{Макросы}
\begin{CompactItemize}
\item 
\#define \hyperlink{application_8hpp_4ddea5668a6504891995f3942304edb6}{RX\_\-SIMB\_\-QUEUE\_\-SZ}~64
\begin{CompactList}\small\item\em длина буфера принимаемых символов. \item\end{CompactList}\item 
\hypertarget{application_8hpp_b9bf261db09b7ffe0b27172004f8bb42}{
\#define \hyperlink{application_8hpp_b9bf261db09b7ffe0b27172004f8bb42}{RX\_\-MESSAGE\_\-QUEUE\_\-SZ}~8}
\label{application_8hpp_b9bf261db09b7ffe0b27172004f8bb42}

\begin{CompactList}\small\item\em длина очереди сообщений. \item\end{CompactList}\end{CompactItemize}
\subsection*{Функции}
\begin{CompactItemize}
\item 
\hyperlink{class_command_parsing_614d78523d2c549667f60ebbb7c8cf0a}{CommandParsing::enCmd\_\-t} \hyperlink{application_8hpp_5f6746b4a57d682f518623bbc7982363}{GetCommand} (char $\ast$rx\_\-message)
\end{CompactItemize}
\subsection*{Переменные}
\begin{CompactItemize}
\item 
\hypertarget{application_8hpp_2cf715bef37f7e8ef385a30974a5f0d5}{
UART\_\-HandleTypeDef \hyperlink{application_8hpp_2cf715bef37f7e8ef385a30974a5f0d5}{huart1}}
\label{application_8hpp_2cf715bef37f7e8ef385a30974a5f0d5}

\begin{CompactList}\small\item\em прототип глобальной струтуры USART1. \item\end{CompactList}\end{CompactItemize}


\subsection{Подробное описание}
: содержит класс, использующий STL, и функцию для распознования команды в строке 

\begin{Desc}
\item[Автор:]: rual \end{Desc}
\begin{Desc}
\item[Дата:]: 18.07.2018 \end{Desc}


\subsection{Макросы}
\hypertarget{application_8hpp_4ddea5668a6504891995f3942304edb6}{
\index{application.hpp@{application.hpp}!RX\_\-SIMB\_\-QUEUE\_\-SZ@{RX\_\-SIMB\_\-QUEUE\_\-SZ}}
\index{RX\_\-SIMB\_\-QUEUE\_\-SZ@{RX\_\-SIMB\_\-QUEUE\_\-SZ}!application.hpp@{application.hpp}}
\subsubsection[{RX\_\-SIMB\_\-QUEUE\_\-SZ}]{\setlength{\rightskip}{0pt plus 5cm}\#define RX\_\-SIMB\_\-QUEUE\_\-SZ~64}}
\label{application_8hpp_4ddea5668a6504891995f3942304edb6}


длина буфера принимаемых символов. 

для использоваания класса map библиотеки STL для дешифровки команды $\ast$ раскоментируйте define USE\_\-STL, код рабочий и компилируется корректно, $\ast$ но на платформе STM32F3Discovery не запускается из-за недостатка RAM $\ast$ В рабочей версии программы применена стандартная библиотека string.h $\ast$ и более легкая функция strcmp() 

\subsection{Функции}
\hypertarget{application_8hpp_5f6746b4a57d682f518623bbc7982363}{
\index{application.hpp@{application.hpp}!GetCommand@{GetCommand}}
\index{GetCommand@{GetCommand}!application.hpp@{application.hpp}}
\subsubsection[{GetCommand}]{\setlength{\rightskip}{0pt plus 5cm}{\bf CommandParsing::enCmd\_\-t} GetCommand (char $\ast$ {\em rx\_\-message})}}
\label{application_8hpp_5f6746b4a57d682f518623bbc7982363}


GetCommand выполняет распознавание команды \begin{Desc}
\item[Аргументы:]
\begin{description}
\item[{\em char$\ast$}]rx\_\-message стандартная строка, содержащая команду \end{description}
\end{Desc}
\begin{Desc}
\item[Возвращает:]одно из значений enCmd\_\-t, соответсвующая команде, либо, если команда не распознана - CMD\_\-NULL \end{Desc}
